\documentclass[10pt,spanish]{article}
\usepackage{graphicx}
\graphicspath{ {images/} }
\usepackage[T1]{fontenc}
\usepackage[utf8]{inputenc}

\usepackage{fancyhdr}
\usepackage{amssymb}
\usepackage{amsmath}
\usepackage{enumerate}

\providecommand{\abs}[1]{\lvert#1\rvert}
\pagestyle{fancy}
\renewcommand{\contentsname}{Índice}


\rhead{TSI}
\lhead{Práctica 3}


\author{
	\includegraphics[scale=5]{UGR} \\\\
	\Large 	Antonio Manuel Fresneda Rodríguez\\
	\\antoniomfr@correo.ugr.es
	\\
}
\date{}
\title{\huge \textbf{Práctica 3\\ Tecnología sistemas inteligentes}}
\begin{document}
	\maketitle
	\pagebreak
	\tableofcontents
	\pagebreak
	\section{Ejercicio 0}
	Este ejercicio comprobamos que el dominio y el problema funcionan correctamente. La salida del planificador esta en el archivo EJ0.txt
	\section{Ejercicio 1}
	Este problema no se puede resolver. Esto ocurre debido a que la tarea de transportar a una persona que no esté en el mismo lugar que el avión no está definida.
	Para resolverlo la hemos implementado he definido un nuevo caso para la tarea.
	\begin{itemize}
		\item \textit {Precondiciones}: Que el avión esté en cA, la persona esté en cP y cp sea distinto de cA.
		\item \textit {Acciones}: El avión va de Ca a Cp, la persona embarca en el avión y el avión vuela a la ciudad de destino del pasajero.
	\end{itemize}
	La salida del plan esta en el archivo EJ1.txt en la carpeta EJ1.
	\section{Ejercicio2}
	En este problema nos dicen que tenemos que tener en cuenta el combustible.\\
	El dominio del ejercicio anterior no tenía en cuenta que nos quedasemos sin combustible. Para ello he añadido \textit{method: fuel-no-suficiente} a la tarea: \textit{mover-avion}.\\
		\begin{itemize}
			\item \textit{Precondiciones}: Que no haya fuel para viajar de la ciudad c1 a c2.
			\item \textit{Acciones}: Recargamos el depósito en la ciudad 1 y volamos despues a la ciudad 2.
		\end{itemize}
	\section{Ejercicio3}
	

\end{document}
