\documentclass[10pt,spanish]{article}
\usepackage{graphicx}
\graphicspath{ {images/} }
\usepackage[T1]{fontenc}
\usepackage[utf8]{inputenc}

\usepackage{fancyhdr}
\usepackage{amssymb}
\usepackage{amsmath}
\usepackage{enumerate}

\providecommand{\abs}[1]{\lvert#1\rvert}
\pagestyle{fancy}
\renewcommand{\contentsname}{Índice}


\rhead{TSI}
\lhead{Práctica 3}


\author{
	\includegraphics[scale=5]{UGR} \\\\
	\Large 	Antonio Manuel Fresneda Rodríguez\\
	\\antoniomfr@correo.ugr.es
	\\
}
\date{}
\title{\huge \textbf{Práctica 3\\ Técnica sistemas inteligentes}}
\begin{document}
	\maketitle
	\pagebreak
	\tableofcontents
	\pagebreak
	\section{Ejercicio 0}
	Este ejercicio comprobamos que el dominio y el problema funcionan correctamente. La salida del planificador esta en el archivo EJ0.txt
	\section{Ejercicio 1}
	Este problema no se puede resolver. Esto ocurre debido a que la tarea de transportar a una persona que no esté en el mismo lugar que el avión no está definida.
	Para resolverlo la hemos implementado he definido un nuevo caso para la tarea.
	\begin{itemize}
		\item \textit {Precondiciones}: Que el avión esté en cA, la persona esté en cP y cp sea distinto de cA.
		\item \textit {Acciones}: El avión va de Ca a Cp, la persona embarca en el avión y el avión vuela a la ciudad de destino del pasajero.
	\end{itemize}
	La salida del plan esta en el archivo EJ1.txt en la carpeta EJ1.
	\section{Ejercicio2}
	En este problema nos dicen que tenemos que tener en cuenta el combustible.\\
	El dominio del ejercicio anterior no tenía en cuenta que nos quedásemos sin combustible. Para ello he añadido \textit{method: fuel-no-suficiente} a la tarea: \textit{mover-avion}.\\
		\begin{itemize}
			\item \textit{Precondiciones}: Que no haya fuel para viajar de la ciudad c1 a c2.
			\item \textit{Acciones}: Recargamos el depósito en la ciudad 1 y volamos después a la ciudad 2.
		\end{itemize}
	La salida del plan está en el archivo EJ2.txt en la carpeta E2.
	\section{Ejercicio3}
	Para poder representar en nuestro dominio que el avión puede tener dos tipos de velocidad (lento y rápida) y dos tipo de consumo (lento y rápido) vamos a tener que modificar los predicados derivados para estimar si vamos a tener fuel y la tarea de moverse del avión, ya que va a recorrer más distancia en menos tiempo.\\
	He declarado la función \textit{(fuel-limit)} y he añadido los predicados \textit{(hay-fuel-viaje-lento ?a ?c1 ?c2)} y \textit{(hay-fuel-viaje-rapido ?a ?c1 ?c2)}\\
	El derivado de cada uno deduce que el el fuel del avión \text{?a} sea mayor que el consumo volado de forma lenta/rápida multiplicado por la distancia entre las ciudades ?c1 y ?c2.\\
	Una vez hecho esto, he modificado la tarea de mover el avión añadiendo lo siguiente:
	\begin{itemize}
		\item \textit{fuel-suficiente-viaje-(lento o rápido)}
		\begin{itemize}
			\item \textit{Precondiciones}: Que el avión tenga suficiente fuel para viajar de ?c1 a ?c2 viajando de forma lenta/rápida y que el total de fuel usado (que se incrementa cada vez que se vuela de un sitio a otro) sea menor que el limite de fuel. 
			\item \textit{Tarea}: Volar de ?c1 a ?c2
		\end{itemize}
		\item \textit{fuel-no-suficiente-viaje-(lento o rápido)}
		\begin{itemize}
			\item \textit{Precondiciones}: Que el avión no tenga suficiente fuel para viajar de ?c1 a ?c2 viajando de forma lenta/rápida y que el total de fuel usado (que se incrementa cada vez que se vuela de un sitio a otro) sea menor que el limite de fuel. 
			\item \textit{Tarea}:Repostar el avión en ?c1 y Volar de ?c1 a ?c2
		\end{itemize}
	\end{itemize}
	La salida del plan está en el archivo EJ3.txt

\end{document}
