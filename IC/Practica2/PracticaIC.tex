\documentclass[10pt,spanish]{article}
\usepackage{graphicx}
\graphicspath{ {images/} }
\usepackage[T1]{fontenc}
\usepackage[utf8]{inputenc}

\usepackage{fancyhdr}
\usepackage{amssymb}
\usepackage{amsmath}
\usepackage{enumerate} 

\providecommand{\abs}[1]{\lvert#1\rvert}
\pagestyle{fancy}
\renewcommand{\contentsname}{Índice}


\rhead{Ingeniería de Conocimiento}
\lhead{Práctica Final}


\author{
	\includegraphics[scale=1.1]{UGR} \\\\
	\Large 	Antonio Manuel Fresneda Rodríguez\\
	\\antoniomfr@correo.ugr.es
	\\77447672-W
	}
\date{}
\title{\huge \textbf{Práctica final Ingeniería de Conocimiento}}
\begin{document}
	\maketitle
	\pagebreak
	\tableofcontents
	\pagebreak
	\section{Ejercicio1}
	Este ejercicio se realizó en la práctica 1. 
	\section{Ejercicio2}
	\begin{itemize}
		\item \textbf{Menos de N empleados TG}\\
		Para contar los empleados que de distintos trámites que tenemos he añadido dos nuevos hechos llamados \textit{EmpleadosLibres ?tramite ?total} que indica los empleados que están libres en ese momento.\\
		También he modificado las reglas del ejercicio 1B para que cada vez que un trabajador pulsa el botón de disponible se incremente este contador y cuando se le asigne un usuario disminuya.
		Una vez hecho esto, he definido una nueva regla que captura el hecho \textit{EmpleadosLibres TG ?tot} y en el caso de que ?tot sea menor que la cantidad mínima de empleados se notifique. Esta regla se llama \textbf{TGLibres}
		\item \textbf{No hay empleados TE}\\
		\item \textbf{Usuario que lleva más de un tiempo 	máximo de espera}\\ Lo que he hecho ha sido que a la hora de que un usuario entre a la cola de espera, calculo el tiempo actual del sistema con la siguiente orden:\\ \textit{(bind ?t (+ (hora-segundos (horasistema)) (minuto-segundos (minutossistema)) (segundo-segundos (segundossistema))))
		}\\
		Se añade un hecho con el identificador del usuario y su tiempo. \\Luego tengo una regla que captura este hecho  (identificando aquellos usuarios cuyo identificador sea mayor al identificador del ultimo usuario atendido de dicho tramite) y si se da la condición:\\ \textit{(if (> (- (+ (hora-segundos (horasistema)) (minuto-segundos (minutossistema)) (segundo-segundos (segundossistema))) ?tiempo) ?tiempoMax)}\\
		Se imprime por pantalla si se cumple la condición que dicho usuario ya lleva demasiado tiempo esperando. Esta regla se llama: \textbf{ComprobarTiempo}
	\item \textbf{Usuario cuyo trámite se retrasa}\\
	\end{itemize}
\end{document}
